%%%%%%%%%%%%%%%%%%%%%%%%%%%%%%%%%%%%%%%%%%%%%%%%%%%%%%%%%
%COPYRIGHT
%%%%%%%%%%%%%%%%%%%%%%%%%%%%%%%%%%%%%%%%%%%%%%%%%%%%%%%%%


    % Copyright (C) 2016 Manuel Castillo López

    % This program is free software: you can redistribute it and/or modify
    % it under the terms of the GNU General Public License as published by
    % the Free Software Foundation, either version 3 of the License, or
    % (at your option) any later version.

    % This program is distributed in the hope that it will be useful,
    % but WITHOUT ANY WARRANTY; without even the implied warranty of
    % MERCHANTABILITY or FITNESS FOR A PARTICULAR PURPOSE.  See the
    % GNU General Public License for more details.

    % You should have received a copy of the GNU General Public License
    % along with this program.  If not, see <http://www.gnu.org/licenses/>.
    
\setcounter{chapter}{1}

\chapter{Estructuras algebraicas}

\section{Operación interna}
Sea $A$ un conjunto no vacío. Se llama operación interna definida en $A$ a cualquier aplicación de $A \times A$ en $A$ que asocia a cada par $(a,b)$ de elementos de $A$ un único elemento $c$, resultado de operar $a$ con $b$. Mátemáticamente, para el operador ''$*$'', se expresa de la siguiente forma:\\

\begin{minipage}{0.5\textwidth}
\begin{center}
$A\times A \overset{*}{\rightarrow} A$\\
$(a,b) \rightarrow c:=a*b$
\end{center}
\end{minipage}
\begin{minipage}{0.5\textwidth}
con $a,b,c \ \in A$
\end{minipage}\\

\subsubsection*{Ejemplo: }
El producto de números reales ($\mathbb{R}, \cdot $) es una operación interna del conjunto de los números reales, ya que cualquier producto de números reales da como resultado otro número real.
$$
x \cdot y \in \mathbb{R} \quad \forall x,y \in \mathbb{R}
$$

\subsection*{Propiedades}
Sea $(A,*)$ un conjunto no vacío ($A$) donde hay definida una operación interna ($*$). Diremos que la operación es:

\begin{itemize}
\item \textbf{Asociativa} $\Leftrightarrow a*(b*c)=(a*b)*c \quad \forall a,b,c \in A$
\item \textbf{Conmutativa} $\Leftrightarrow a*b=b*a \quad \forall a,b \in A$
\end{itemize}

Añadamos una nueva operación interna a nuestro par, obteniendo $(A,*,\circ)$. Diremos que $\circ$ es \textbf{distributiva} respecto de $*$ si
$$
\forall a,b,c \in A, 
\begin{cases}
a\circ (b*c)=(a\circ b)*(a\circ c),\\
(a*b)\circ c= (a \circ c) * (b \circ c)
\end{cases}
$$

\subsection*{Elementos particulares}
\begin{itemize}
\item Elemento neutro $e$:\quad $a*e=e*a=a$
\item Elemento simétrico $a'$:\quad $a*a'=a'*a=e$
\end{itemize}

\subsubsection*{Ejemplo:}
Sea $(\mathbb{R},+,\cdot )$ el conjunto de los números reales con las operaciones internas de producto y suma. Podemos comprobar que tanto el producto como la suma son asociativos y conmutativos. Además el producto es distributivo respecto de la suma ya que:
$$
\forall a,b,c \in \mathbb{R}, 
\begin{cases}
a\cdot (b+c)=(a\cdot b)+(a \cdot c),\\
(a+b)\cdot c= (a \cdot c) + (b \cdot c)
\end{cases}
$$


\section{Operación externa}
Dados dos conjuntos $A$ y $K$, se llama operación externa definida en A y con dominio de escalares $K$, a cualquier aplicación:\\

\begin{minipage}{0.5\textwidth}
\begin{center}
$K\times A \overset{\bot}{\rightarrow} A$\\
$(k,a) \rightarrow b:=k\bot a$
\end{center}
\end{minipage} con
\quad
\begin{minipage}{0.4\textwidth}
$k\in K$\\
$a\in A$
\end{minipage}\\

\subsubsection*{Ejemplo:}

Sea $V_3 \equiv \lbrace v=(x,y,z)\quad \forall x,y,z \in \mathbb{R}\rbrace$ el conjunto de los vectores reales de tres dimensiones y el conjunto de los escalares enteros  $K\equiv \lbrace k\quad \forall k \in \mathbb{Z}\rbrace$. El producto de escalares por vectores es operación externa ya que $k\cdot v \in V_3$.

\section{Homomorfismos}
Sean $(A,*)$ y $(B,\circ)$ dos conjuntos con operaciones internas definidas. Una aplicación $f: A \rightarrow B$ es un \textbf{homomorfismo} si
$$
f(a*b)=f(a)\circ f(b), \quad \forall a,b \in A
$$

Si $f$ es un homomorfismo y además
\begin{itemize}
\item es inyectivo se llamará \textbf{monomorfismo}.
\item es sobreyectivo se llamará \textbf{epimorfismo}.
\item es biyectivo se llamará \textbf{isomorfismo}.
\item $A=B$ se llamará \textbf{endomorfismo}.
\item es endomorfismo biyectivo se llamará \textbf{automorfismo}
\end{itemize}



\subsubsection*{Ejemplo:}
Sean $G=(\mathbb{R},+)$ y $H=(\mathbb{R}^+,\cdot)$. Definamos una aplicación
$$
\begin{matrix}
f:G\leftarrow H\\
x \mapsto e^x
\end{matrix}
$$

Podemos afirmar que se trata de un homomorfismo ya que

$$
\begin{matrix}
f(x+y)=f(x)\cdot f(y)\\
e^{x+y}=e^x+e^y
\end{matrix}
$$
\section{Grupo}
Un \textbf{grupo} es una pareja $(G,*)$, donde G es un conjunto en el que está definida una operación interna $*$ que verifica:
\begin{enumerate}
\item Asociativa.
\item Existencia de elemento neutro $e$, es decir, $g*e=g$
\item Todo elemento $g$ posee simétrico $g'$, es decir, $g*g'=e$
\end{enumerate}
Si además la operación interna $*$ es conmutativa, el grupo se llamara \textbf{abeliano}.\\

\subsection*{Propiedades de un grupo}
\begin{itemize}
\item El elemento neutro es único
\item $(a*b)'=b'*a'$
\item $(a')'=a$
\item $a*x=a*y \Rightarrow x=y$
\end{itemize}

\subsection*{Orden de un grupo}
Sea $(G,*)$ un grupo. El \textbf{orden} de un elemento $a \in G$ es el menor entero positivo $k\in \mathbb{N}^*$ para el que $a^k=e$. Si no existe $k$, el orden es infinito o cero.\\

Al número de elementos de un grupo se le llama \textbf{orden} de $G$ y se denota por $|G|$.\\

%Si $(G,*)$ sólo cumple la propiedad asociativa será un \textbf{semigrupo}. Si además posee elemento neutro hablaremos de un \textbf{monoide}.


\subsubsection*{Ejemplo:}

El conjunto de los números enteros con la suma $(\mathbb{Z},+)$ es un grupo abeliano ya que:
\begin{enumerate}
\item La suma de numeros enteros es otro número entero.
\item El elemento neutro de los enteros con la suma es el cero: $z+0=z$.
\item Todo $z$ posee un simétrico $-z$: $z+(-z)=0$.
\item La suma de enteros es conmutativa: $z_1+z_2=z_2+z_1$.
\end{enumerate}

A modo de ejemplo diremos que el orden del grupo es infinito $|\mathbb{Z}|=\infty$. El orden de los elementos son

\begin{flalign*}
&|1|=1\\
&|-1|=2\\
&|z|=\infty \quad \forall z \in \mathbb{Z}-\lbrace-1,1\rbrace
\end{flalign*}

\subsection*{Subgrupo}
Sea $(G,*)$ un grupo y $H\subset G$, un subconjunto suyo no vacío. $(H,*)$ es \textbf{subgrupo}  si también posee estructura de grupo.

\subsubsection*{Caracterización}
Un subconjunto $H$ es subgrupo si se cumple que

\begin{flalign}
&\forall a,b \in H \Rightarrow a*b \in H
\label{sub1}\\
&\forall a \in H \Rightarrow a' \in H
\label{sub2}
\end{flalign}

Las ecuaciones \ref{sub1} y \ref{sub2} se pueden resumir en la siguiente

$$
\forall a,b \in H \Rightarrow a*b' \in H
$$

\subsection*{Clases de un grupo}
Dado un grupo $G$, un subgrupo $H$ y un elemento $a\in G$ arbitrario fijo, a los conjuntos

\begin{align*}
aH=\lbrace x/x\in G, x=a*h, h\in H \rbrace\\
Ha=\lbrace x/x\in G, x=h*a, h\in H \rbrace
\end{align*}

se les denomina respectivamente \textbf{clases} del grupo G a la izquierda y a la derecha módulo el subgrupo $H$. Se les llama así por ser clases de cierta relación de equivalencia \footnote{La relación es $x\sim y \Leftrightarrow x'*y \in H$, aunque no es de interés para ésta explicación.} y, por tanto, forman particiones del grupo $G$.\\

Si $aH=Ha$ entonces, $H$ es un subgrupo \textbf{normal o invariante}.\\

Supongamos que el grupo $G$ es finito y que posee $n$ clases a la izquierda módulo $H$. Entonces,
$$
G=a_1H \cup a_2H \cup \dots \cup a_nH
$$
$$
|G|=|a_1H|+|a_2H|+ \dots +|a_nH|=n|H|
$$

por lo que el orden de un grupo finito $G$ será múltiplo del orden de cualquier subgrupo suyo (\textbf{teorema de Lagrange}).\\

Al cociente $n=|G|/|H|$ se le denomina \textbf{índice del subgrupo} $H$.\\

Un grupo $G$ se dice \textbf{finitamente generado} si existe una parte finita $A$ de $G$ que engendra todo $G$. Si $A$ se reduce a un elemento, el grupo $G$ se llama \textbf{monógeno}.\\

El grupo $G$ es \textbf{cíclico} si es monógeno y finito.

$$
G=<a>=\lbrace a^n : n \in \mathbb{Z} \rbrace
$$

\emph{Nota:} Con $a^n$ nos referimos a aplicar $n$ veces el operador $*$ sobre $a$. Ésto coincidirá con la potencia en el caso de que la operación sea el producto pero, en general, no es así.

\subsection*{Homomorfismo de grupos}
Al igual que en la sección anterior, dos grupos $(G_1,*)$, $(G_2,\circ)$ y una aplicación $f:G_1 \rightarrow G_2$ es \textbf{homomorfismo} si 

$$
f(a*b)=f(a)\circ f(b), \quad \forall a,b \in G_1
$$

Llamamos \textbf{núcleo} de $f$, representándose por $Ker(f)$, al conjunto de los elementos del dominio cuya imagen es el elemento neutro de $G_2$.

$$
Ker(f)=\lbrace x\in G_1 : f(x)=e_2 \rbrace = f^{-1}(e_2)
$$

Llamamos \textbf{imagen} de $f$, denotándose por $Im(f)$, como el subconjunto de $G_2$ formado por aquellos elementos que son imagen de algún elemento de $G_1$. Es decir,

$$
Im(f)= \lbrace y \in G_2 : \exists x \in G_1, f(x)=y \rbrace
$$

Por tanto, podemos decir que 
\begin{flalign*}
& f\ es\ inyectiva\ \Leftrightarrow Ker(f)=\lbrace e_1\rbrace\\
& f\ es\ sobreyectiva\ \Leftrightarrow Im(f)=G_2
\end{flalign*}
\section{Anillo}
Un anillo es un conjunto dotado con dos operaciones internas llamadas suma y producto. El anillo $(R,+,\cdot)$ cumple que:
\begin{enumerate}
\item $(R,+)$ es un grupo abeliano.
\item El producto es asociativo.
\item Existe un elemento neutro para la multiplicación.
\item El producto es distributivo respecto a la suma.
\end{enumerate}
Si el producto es conmutativo se dice que el anillo es conmutativo. Si el producto posee elemento neutro es unitario.\\

El elemento neutro de la suma será 0 y el del producto 1.

\subsection*{Subanillo}

\section*{Cuerpo}
Un cuerpo es un anillo conmutativo y unitario en el que todo elemento distinto de cero es invertible respecto del producto, es decir, un anillo de división conmutativo.\\

Por ejemplo, los números reales con la suma y el producto algebraicos $(\mathbb{R},+,\cdot)$ es un cuerpo ya que:
\begin{enumerate}
\item $(\mathbb{R}, +)$ es un grupo abeliano (siendo 0 el elemento neutro de la suma)
\item El producto de números reales es asociativo.
\item El elemento neutro de la multiplicación es el 1.
\item El producto es distributivo respecto de la suma (propiedades de anillo cumplidas).
\item El anillo es conmutativo, ya que el producto de números reales lo es.
\item El anillo es unitario ya que el neutro de la multiplicación es distinto del de la suma.
\item Todo elemento distinto de cero es invertible respecto del producto:
Sea $r\in \mathbb{R}$ y $r \neq 0$, entonces $\frac{1}{r} \in \mathbb{R}$.
\end{enumerate}
Así lo serían también $(\mathbb{C},+,\cdot)$, $(\mathbb{Z},+,\cdot)$ y $(\mathbb{Q},+,\cdot)$.

