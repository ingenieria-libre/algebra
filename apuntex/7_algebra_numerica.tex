%%%%%%%%%%%%%%%%%%%%%%%%%%%%%%%%%%%%%%%%%%%%%%%%%%%%%%%%%
%COPYRIGHT
%%%%%%%%%%%%%%%%%%%%%%%%%%%%%%%%%%%%%%%%%%%%%%%%%%%%%%%%%


    % Copyright (C) 2016 Manuel Castillo López

    % This program is free software: you can redistribute it and/or modify
    % it under the terms of the GNU General Public License as published by
    % the Free Software Foundation, either version 3 of the License, or
    % (at your option) any later version.

    % This program is distributed in the hope that it will be useful,
    % but WITHOUT ANY WARRANTY; without even the implied warranty of
    % MERCHANTABILITY or FITNESS FOR A PARTICULAR PURPOSE.  See the
    % GNU General Public License for more details.

    % You should have received a copy of the GNU General Public License
    % along with this program.  If not, see <http://www.gnu.org/licenses/>.
\documentclass[0_algebra.tex]{subfiles}
\begin{document}

\onlyinsubfile{
\title{\centering \huge \bfseries Álgebra Lineal}
\author{
	\emph{ \bfseries \centering Tema 7: Álgebra Numérica}\\[1.5 cm]
 %	\emph{\centering Open Source Project}\\[1.5 cm]
 % 	\textsc{Escuela Técnica Superior de Ingeniería Industrial}\\
 % 	\textsc{University of Málaga}\\[0.5 cm]
 	\texttt{ingenierocontracabrones.blogspot.com}\\[0.5 cm]
	\texttt{mclnaranjito@gmail.com}\\[4 cm]
	\small{Copyright \copyright \ (2016 - \the\year) \ Manuel Castillo López.} \\
	\small{GPL GNU General Public License}\\[0.5 cm]
	}
\maketitle
\tableofcontents

\setcounter{chapter}{6} 
}

\chapter{Álgebra lineal numérica}
\section{Normas matriciales}
\section{Métodos de Jacobi y Gauss-Seidel}
\section{Factorización de matrices LU y QR}
\section{Estimación de errores}
\section{Implementación de algoritmos}
\section{Cálculo de autovalores y autovectores}


\end{document}
